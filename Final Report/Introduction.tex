\chapter{Introduction}

Today's clothing industry is the most multitudinous and developed than ever before in history, offering consumers almost endless choices. However, a significant problem that plagues the industry is the lack of a standardised sizing scheme for clothes. 

Majority of consumers do not know their exact body measurements or parameters as most current methods for obtaining this information are inaccessible. This has resulted in the status quo, where determining the correct sizes of clothing to buy is a process of physically travelling to a clothing retailer, trying on various items and making a judgement call, based on the fit. This process is exacerbated by the fact that a consumer's body measurements are prone to variation and the inconsistency in sizes across different brands. For these reasons, clothing shopping is often very laborious and frustrating for many consumers.

Also producer side - Waste etc.
Presence of online and augmented reality and personalisation

\section{Background to the study}
A very brief background to your area of research. Start off with a general introduction to the area and
then narrow it down to your focus area. Used to set the scene \cite{smt2011}.
\section{Objectives of this study}
\subsection{Problems to be investigated}
Description of the main questions to be investigated in this study.
\subsection{Purpose of the study}
Give the significance of investigating these problems. It must be obvious why you are doing this study
and why it is relevant.

\section{Scope and Limitations}
Scope indicates to the reader what has and has not been included in the study. Limitations tell the
reader what factors influenced the study such as sample size, time etc. It is not a section for excuses as
to why your project may or may not have worked.

\section{Plan of development}
Here you tell the reader how your report has been organised and what is included in each
chapter.

{\bf I recommend that you write this section last. You can then tailor it to your report.}


