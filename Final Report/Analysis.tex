\chapter{Analysis} \label{analysis}

Mention data set?\\

The observed factors that had the greatest contribution to inaccuracy are listed below in order of impact:

\begin{enumerate}
	\item The system not being able to accurately determine the extremities of the body.
	\item The inconsistency of the plane being used to take the measurement.
	\item The error or "jitter" of the tracked skeleton.
\end{enumerate}

Therefore, the system performed such that it met the requirements of achieving an accuracy better than 25\%, as stipulated in section \ref*{methodologyAim}.

The results obtained in Table \ref{tab:viewResults}, together with the detailed results of each volunteer available in Appendix \ref{appendixDetailedResults} have yielded insights for each of the views. They are discussed in the subsequent subsections. 

\subsection{Front Performance}

The "Front" view performed the best overall with an average accuracy of 20.67\%. This can be attributed to the fact that the Kinect performs the best when a person faces it head on (Parallel to the image plane). This is due to the Kinect being able to fully track the skeleton of the user. As a result, a more accurate and reliable skeletal coordinate system can be used, which in turn provide more accurate planes of measure. 

All joints and skeleton tracked 

\subsection{Left Performance}

\subsection{Back Performance}

The "Back" view performed better than expected despite the average error of 23.11\%

\subsection{Right Performance}

Front was the most accurate

Misc\\
Missing - Height, Skinny, Back, Skeleton Mapping
8) Empirical Insights - Trying to determine a correct measurement,  length of people, skinny person, skeleton not perfectly fitting, background edges not perfect \\

5) Uncertainty model\\