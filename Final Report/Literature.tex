\chapter{Literature Review}

Technology background\\
1) Depth Sensor technology + How Kinect works\\
2) Point cloud map\\

Coding references/Getting started\\
3) Code references and and blog posts?\\ 
4) Previous example\\
5) Hand Example\\

Mathematics Used\\
4) Papers on ellipse circumference\\

Improving Accuracy\\
5) Skeleton Joints filtering\\

Further developments\\
6) Augmented reality paper\\

Imaging Processing Background\\
7) Basics of an image - RGB\\
8) Matlab\\
9) Camera Model\\


Once upon a time engineers and researchers believed... In this area of research, they used the following methods... \cite{jct2010}

Write this section first as it will take you the longest. I suggest you start writing this as soon as you
have done your initial research at the beginning of your project. You can then return to it once you
have completed your work to edit and adjust it.

A literature review forms the theoretical basis of your project. You need to read a large number of
journal papers, sections in books, technical reports etc. relevant to your work at the start of project.
This will give you a good idea of the field of research.

When writing your review start of with the general concepts and move to the more specific aspects
explaining the necessary theory as you go. This section is NOT a copy and paste from others work or a
rewrite-but-change-one-word section. I suggest you read all your material, and then put it down and
write this section, referring back to the work only when you need to check something.

See your PCS textbook for more details on how to write a literature review.

If you include a figure or a table in your text please see the example in Fig. \ref{fig:model} as to how to caption it.
Please make sure that all text in your figures is readable and that you reference your figures if they are
from another source.

\begin{figure}[ht]
\centering
\includegraphics[width=0.7\textwidth]{model.png}
\caption{A block diagram illustrating the connections to the IRQ pin on the MCS08GT16A microcontroller (Please
note that your headings should be short descriptions of what is in the diagram not simply the figure title)}
\label{fig:model}
\end{figure}

