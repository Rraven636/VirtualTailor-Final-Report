\chapter{Solution Design}

\section{Implementation Design}


The key aim of the overall solution is to alleviate stresses on both the consumer and supplier side of : Reduce inconvenience experienced by consumers and limit resources wasted by producers.

The solution will achieve this in the following key manners:
\begin{itemize}
	\item Create a an easy to use sysetm
\end{itemize} 

This overall solution aims to be applicable and provide relief in all three channels clothing channels mentioned in Section \ref{clothingIndustryContext}. However, in order to do this, an analysis of pain points of consumers and producers in each channel must be performed. Shown in Table \ref{tab:conProdPainPoints} is a summary of such an analysis.

\begin{table}[htbp]
	\centering
	\caption{Pain points of consumers and producers in each of the three channels}
	\begin{tabularx}{\textwidth}{|Y|Y|Y|}
		\toprule
		& 
		\textit{\textbf{Consumers}} & 
		\textit{\textbf{Producers}} \\
		\midrule
		\textit{\textbf{Retail}} & 
		\begin{itemize}
			\item Visiting stores is time consuming
			\item Trying clothes on is tedious
			\item The correct size is often not available 
		\end{itemize} & 
		\begin{itemize}
			\item Dressing rooms waste valuable space
			\item Trying on clothes pose a security risk and require resources to control
			\item Significant costs are incurred during exchanges 
		\end{itemize} \\
		\midrule
		\textit{\textbf{Online}} & 
		\begin{itemize}
			\item Difficult to choose the correct size
			\item Not sure how the item will look on body
			\item Frustrating to exchange if item is incorrect
		\end{itemize} & 
		\begin{itemize}
			\item Exchanges severely impact their profitability
			\item Experience lower sale volume due as they cannot provide a complete customer experience
		\end{itemize}\\
		\midrule
		\textit{\textbf{Tailoring}} & 
		\begin{itemize}
			\item Only available for specialised clothing
			\item The process is time consuming
			\item There is always delay between getting measured and picking up clothing
		\end{itemize} & 
		\begin{itemize}
			\item Time required to complete measurements limits productivity
			\item Measurements can only take place in person with a trained tailor
		\end{itemize} \\
		\bottomrule
	\end{tabularx}%
	\label{tab:conProdPainPoints}%
\end{table}%

1) Online profile of people - Used for online shopping and retail shops\\
2) Take measurements at a retailer - Virtual Dressing room - A part of the shopping experience and will reduce hassle of trying on clothes\\
3) Amount wasted in trying on clothes or online returns?\\
4) Could be used for personalised tailoring
5) Example of UI - Explanation of how it works

\section{Component Selection}
1) Choice of Kinect - Compare to other depth sensors\\
- Kinect v1 vs v2
- Winddows vs Xbox
2) Choice of Windows SDK\\

\section{Algorithm Design}
1) Windows examples used - Background Removal, Colour Stream and Skeleton Tracking - NB - Why BackgroundRemoval instead of own method
3D Points - No calibration

2) Run through of algorithm
- Background Removed frame
- Send image to separate class for processing
- Create array with background removed pixels
- Draw skeleton on image
- Create axes for measurement - Perpendicular or straight depending on particular measurement

\section{Experimental Design}
- Constraints - Men, distance from Kinect, height of Kinect, Number of views, 3D Modelling, control distance - box of 0.5m
- UI to run simulated dressing room
- Volunteer to pose as instructed by person controlling UI
- Take measurement of front
- Take left
- Take back
- Take right 
- At each point, take actual readings with uncertainty
- Note: Did not use correction in \cite{nonContact2017}
- For one volunteer, take 5 readings in relatively the same pose - Determine uncertainty 