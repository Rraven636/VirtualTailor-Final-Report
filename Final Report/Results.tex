\chapter{Results}
These are the results obtained from the investigation outlined in \ref{methodology}. Seven volunteers were used in determining the accuracy of the system. They were first measured by the system and then their physical measurements were obtained for comparison. This chapter explores the results obtained for the measurement of the extremities of a human body and the modelling of 3D body parts, together with the observed performance of the user interface respectively.

\section{Length and Extremity Measurement}

This section begins with a presentation of the overall results of the system and a comparison with the aim of the investigation. Subsequent subsections present further performance of specific areas of the system that form the basis of analysis presented in Chapter \ref{analysis}. These sections focus on the following major areas of the system:

\begin{itemize}
	\item The accuracy of each view of measurement (Front, Left, Back and Right).
	\item The accuracy of measuring each individual limb.
	\item The impact that clothing worn by a person being measured has on the system's accuracy.
	\item Other empirical insights obtained through use and observation of the system. 
\end{itemize}


\subsection{Overall Results}
Below is a summary of the aggregate accuracy of the system and the accuracy per volunteer, together with their personal characteristics. (Table \ref{tab:overallAccuracy})
 
 % Table generated by Excel2LaTeX from sheet 'Overall'
 \begin{table}[htbp]
 	\centering
 	\caption{Overall results of accuracy of system per volunteer}
 	\begin{tabularx}{\textwidth}{|YY|YYY|}
 		\toprule
 		\multicolumn{1}{|Y|}{\textit{\textbf{Volunteer Number}}} & \textit{\textbf{Average Error}} & \multicolumn{1}{Y|}{\textit{\textbf{Build}}} & \multicolumn{1}{Y|}{\textit{\textbf{Height}}} & \textit{\textbf{Clothing}} \\
 		\midrule
 		\multicolumn{1}{|Y|}{\textit{\textbf{1}}} & 28.44\% & \multicolumn{1}{Y|}{Athletic} & \multicolumn{1}{Y|}{Tall} & Tight \\
 		\midrule
 		\multicolumn{1}{|Y|}{\textit{\textbf{2}}} & 17.27\% & \multicolumn{1}{Y|}{Athletic} & \multicolumn{1}{Y|}{Average} & Tight \\
 		\midrule
 		\multicolumn{1}{|Y|}{\textit{\textbf{3}}} & 16.45\% & \multicolumn{1}{Y|}{Athletic} & \multicolumn{1}{Y|}{Tall} & Vest \\
 		\midrule
 		\multicolumn{1}{|Y|}{\textit{\textbf{4}}} & 40.75\% & \multicolumn{1}{Y|}{Slim} & \multicolumn{1}{Y|}{Average} & Loose \\
 		\midrule
 		\multicolumn{1}{|Y|}{\textit{\textbf{5}}} & 20.72\% & \multicolumn{1}{Y|}{Big} & \multicolumn{1}{Y|}{Average} & Loose \\
 		\midrule
 		\multicolumn{1}{|Y|}{\textit{\textbf{6}}} & 19.60\% & \multicolumn{1}{Y|}{Slim} & \multicolumn{1}{Y|}{Short} & Tight \\
 		\midrule
 		\multicolumn{1}{|Y|}{\textit{\textbf{7}}} & 18.83\% & \multicolumn{1}{Y|}{Big} & \multicolumn{1}{Y|}{Average} & Vest \\
 		\midrule
 		\multicolumn{5}{|Y|}{} \\
 		\midrule
 		\multicolumn{2}{|V|}{\textit{\textbf{Total Average Error}}} & \multicolumn{3}{W|}{23.15\%} \\
 		\bottomrule
 	\end{tabularx}%
 	\label{tab:overallAccuracy}%
 \end{table}%
 
As seen in Table \ref{tab:overallAccuracy}, despite the average error of some individual volunteers being outside the desired range of 25\% accuracy, the total average error of the system is 23.15\%. 

\subsection{View Performance}

Each "view" (Front, Left, Back or Right) of the system has a unique set of characteristics. It is useful to investigate the performance of each of them to better understand their effectiveness in the system as a whole. 

Seen below in Table \ref{tab:viewResults} are the results of the average accuracy of each view obtained after measuring the volunteers in each of the respective views. 

% Table generated by Excel2LaTeX from sheet 'Overall'
\begin{table}[htbp]
	\centering
	\caption{Results of the average accuracy of each view per volunteer}
	\begin{tabularx}{\textwidth}{|Y|Y|Y|Y|Y|}
		\toprule
		\textit{\textbf{Volunteer Number}} & \textit{\textbf{Front}} & \textit{\textbf{Left}} & \textit{\textbf{Back}} & \textit{\textbf{Right}} \\
		\midrule
		\textit{\textbf{1}} & 21.82\% & 29.90\% & 32.61\% & 31.04\% \\
		\midrule
		\textit{\textbf{2}} & 16.58\% & 15.81\% & 20.45\% & 14.60\% \\
		\midrule
		\textit{\textbf{3}} & 17.27\% & 19.01\% & 15.53\% & 13.37\% \\
		\midrule
		\textit{\textbf{4}} & 35.58\% & 58.15\% & 33.58\% & 44.77\% \\
		\midrule
		\textit{\textbf{5}} & 21.06\% & 10.39\% & 24.92\% & 23.47\% \\
		\midrule
		\textit{\textbf{6}} & 11.65\% & 44.58\% & 14.09\% & 17.07\% \\
		\midrule
		\textit{\textbf{7}} & 20.72\% & 19.51\% & 20.60\% & 12.06\% \\
		\midrule
		\textit{\textbf{Total Avg Error}} & \textit{\textbf{20.67\%}} & \textit{\textbf{28.19\%}} & \textit{\textbf{23.11\%}} & \textit{\textbf{22.34\%}} \\
		\bottomrule
	\end{tabularx}%
	\label{tab:viewResults}%
\end{table}%

As seen in Table \ref{tab:viewResults}, the accuracy of the views in descending order are as follows:

\begin{enumerate}
	\item Front
	\item Right
	\item Back
	\item Left
\end{enumerate}

The only view that performed outside the desired accuracy range of 25\% was the "Left" view with an average accuracy of 28.19\%. The Front view performed the best with an accuracy of 20.67\%.  

\subsection{Limb Performance}
Each extremity being measured is subtly different due to various factors such as position, orientation and size. Therefore, understanding how the system performs in these different cases is useful.

Due to the large number of measurements being taken, the results have been split up in terms of upper and lower body measurements.
 
Below is a summary of the aggregate accuracy of the system in measuring upper body limbs per volunteer: (Table \ref{tab:upperBodyResults})

% Table generated by Excel2LaTeX from sheet 'Overall'
\begin{table}[htbp]
	\centering
	\caption{Results of the average accuracy of Upper Body Limbs}
	\begin{tabularx}{\textwidth}{|Y|Y|Y|Y|Y|Y|}
		\toprule
		\textit{\textbf{Volunteer Number}} & \textit{\textbf{Chest}} & \textit{\textbf{Upper Left Arm}} & \textit{\textbf{Lower Left Arm}} & \textit{\textbf{Upper Right Arm}} & \textit{\textbf{Lower Right Arm}} \\
		\midrule
		\textit{\textbf{1}} & 49.50\% & 29.90\% & 10.10\% & 26.35\% & 11.02\% \\
		\midrule
		\textit{\textbf{2}} & 27.84\% & 23.16\% & 9.84\% & 17.74\% & 13.24\% \\
		\midrule
		\textit{\textbf{3}} & 14.58\% & 10.99\% & 24.14\% & 7.33\% & 12.27\% \\
		\midrule
		\textit{\textbf{4}} & 79.55\% & 12.20\% & 26.74\% & 33.79\% & 5.47\% \\
		\midrule
		\textit{\textbf{5}} & 45.85\% & 21.22\% & 8.41\% & 15.16\% & 14.82\% \\
		\midrule
		\textit{\textbf{6}} & 23.39\% & 6.64\% & 15.91\% & 8.46\% & 8.63\% \\
		\midrule
		\textit{\textbf{7}} & 30.70\% & 12.34\% & 12.71\% & 12.02\% & 7.95\% \\
		\midrule
		\textit{\textbf{Total Avg Error}} & \textit{\textbf{38.77\%}} & \textit{\textbf{16.64\%}} & \textit{\textbf{15.41\%}} & \textit{\textbf{17.26\%}} & \textit{\textbf{10.48\%}} \\
		\bottomrule
	\end{tabularx}%
	\label{tab:upperBodyResults}%
\end{table}%

As seen in Table \ref{tab:upperBodyResults}, measurements of the lower and upper portions of both arms fell well within the desired accuracy of 25\%. However, the chest measurements seemed to be significantly more inaccurate with an average of 38.77\%, which is far outside the desired range.

Below is a summary of the aggregate accuracy of the system in measuring lower body limbs per volunteer: (Table \ref{tab:lowerBodyResults})

% Table generated by Excel2LaTeX from sheet 'Overall'
\begin{table}[htbp]
	\centering
	\caption{Results of the average accuracy of Lower Body Limbs}
	\begin{tabularx}{\textwidth}{|Y|Y|Y|Y|Y|Y|}
		\toprule
		\textit{\textbf{Volunteer Number}} & \textit{\textbf{Waist}} & \textit{\textbf{Upper Left Leg}} & \textit{\textbf{Lower Left Leg}} & \textit{\textbf{Upper Right Leg}} & \textit{\textbf{Lower Right Leg}} \\
		\midrule
		\textit{\textbf{1}} & 32.09\% & 31.47\% & 13.58\% & 28.07\% & 44.04\% \\
		\midrule
		\textit{\textbf{2}} & 25.36\% & 15.32\% & 14.18\% & 13.21\% & 6.61\% \\
		\midrule
		\textit{\textbf{3}} & 20.51\% & 13.09\% & 24.62\% & 20.01\% & 10.12\% \\
		\midrule
		\textit{\textbf{4}} & 80.55\% & 13.36\% & 27.94\% & 25.40\% & 48.15\% \\
		\midrule
		\textit{\textbf{5}} & 26.74\% & 16.08\% & 8.56\% & 24.17\% & 15.81\% \\
		\midrule
		\textit{\textbf{6}} & 35.04\% & 20.78\% & 43.78\% & 9.58\% & 17.38\% \\
		\midrule
		\textit{\textbf{7}} & 19.53\% & 23.78\% & 22.77\% & 26.17\% & 16.17\% \\
		\midrule
		\textit{\textbf{Total Avg Error}} & \textit{\textbf{34.26\%}} & \textit{\textbf{19.13\%}} & \textit{\textbf{22.20\%}} & \textit{\textbf{20.95\%}} & \textit{\textbf{22.61\%}} \\
		\bottomrule
	\end{tabularx}%
	\label{tab:lowerBodyResults}%
\end{table}%

As seen in Table \ref{tab:lowerBodyResults}, measurements of the lower and upper portions of both legs fell within the desired accuracy of 25\%. This was similar to the behaviour of the arm measurements, mentioned above. However, they seemed to be slightly more inaccurate with all of the leg measurements having an accuracy tolerance of more than 19\%, whereas the arms all fell within 17.5\%. The waist measurement also followed the behaviour of the chest measurement above, where seemed to be significantly more inaccurate than the leg measurements. It had an average accuracy tolerance of 34.26\%, which is also outside the desired range.

\subsection{Impact of Clothing}

The nature of the system is such that it is sensitive to clothing worn by the measured party. One aim of the system, as stipulated, in section \ref{methodologyHypothesis}, was to make the system more robust to clothing, such that the results are less prone to error.

As such, each of the above measurement results tables have been adjusted to remove the effects of clothing. This has been done by analysing the data set after volunteers with observed "loose" clothing were removed.

Below is the clothing adjusted version of Table \ref{tab:overallAccuracy}. 

% Table generated by Excel2LaTeX from sheet 'Overall'
\begin{table}[htbp]
	\centering
	\caption{Overall results of accuracy of system per volunteer after adjustments for clothing}
	\begin{tabularx}{\textwidth}{|YY|YYY|}
		\toprule
		\multicolumn{1}{|Y|}{\textit{\textbf{Volunteer Number}}} & \textit{\textbf{Average Error}} & \multicolumn{1}{Y|}{\textit{\textbf{Build}}} & \multicolumn{1}{Y|}{\textit{\textbf{Height}}} & \textit{\textbf{Clothing}} \\
		\midrule
		\multicolumn{1}{|Y|}{\textit{\textbf{2}}} & 17.27\% & \multicolumn{1}{Y|}{Athletic} & \multicolumn{1}{Y|}{Average} & Tight \\
		\midrule
		\multicolumn{1}{|Y|}{\textit{\textbf{3}}} & 16.45\% & \multicolumn{1}{Y|}{Athletic} & \multicolumn{1}{Y|}{Tall} & Vest \\
		\midrule
		\multicolumn{1}{|Y|}{\textit{\textbf{6}}} & 19.60\% & \multicolumn{1}{Y|}{Slim} & \multicolumn{1}{c|}{Short} & Tight \\
		\midrule
		\multicolumn{1}{|Y|}{\textit{\textbf{7}}} & 18.83\% & \multicolumn{1}{Y|}{Big} & \multicolumn{1}{c|}{Average} & Vest \\
		\midrule
		\multicolumn{5}{|Y|}{} \\
		\midrule
		\multicolumn{2}{|V|}{\textit{\textbf{Total Error Average}}} & \multicolumn{3}{W|}{\textit{\textbf{18.04\%}}} \\
		\bottomrule
	\end{tabularx}%
	\label{tab:overallResultsClothingAdj}%
\end{table}%

It is clear from Table \ref{tab:overallResultsClothingAdj} that the system performs significantly better when loose clothing is not worn by the volunteer. The new overall average accuracy of the system improved to 18.04\%, which is well within the desired range of 25\%. 

The same behaviour can be observed for the accuracy of the views and the different limbs.

In the adjusted version of Table \ref{tab:viewResults} (Table \ref{tab:viewResultsClothingAdj}), the performance of all the views improve. Additionally, the "Left" view now also falls within the desired 25\% range and all the other views fall within 18\%. 

% Table generated by Excel2LaTeX from sheet 'Overall'
\begin{table}[htbp]
	\centering
	\caption{Results of the average accuracy of each view per volunteer after adjustments for clothing}
	\begin{tabularx}{\textwidth}{|Y|Y|Y|Y|Y|}
		\toprule
		\textit{\textbf{Volunteer Number}} & \textit{\textbf{Front}} & \textit{\textbf{Left}} & \textit{\textbf{Back}} & \textit{\textbf{Right}} \\
		\midrule
		\textit{\textbf{2}} & 16.58\% & 15.81\% & 20.45\% & 14.60\% \\
		\midrule
		\textit{\textbf{3}} & 17.27\% & 19.01\% & 15.53\% & 13.37\% \\
		\midrule
		\textit{\textbf{6}} & 11.65\% & 44.58\% & 14.09\% & 17.07\% \\
		\midrule
		\textit{\textbf{7}} & 20.72\% & 19.51\% & 20.60\% & 12.06\% \\
		\midrule
		\textit{\textbf{Total Avg Error}} & \textit{\textbf{16.55\%}} & \textit{\textbf{24.73\%}} & \textit{\textbf{17.67\%}} & \textit{\textbf{14.27\%}} \\
		\bottomrule
	\end{tabularx}%
	\label{tab:viewResultsClothingAdj}%
\end{table}%

As for the adjusted upper and lower body measurements (Table \ref{tab:upperBodyResultsClothingAdj} and Table  \ref{tab:lowerBodyResultsClothingAdj} respectively), all of the measurements, except for the "Lower Left Leg" measurement (increased to 26.34\% which is outside the desired range) either improved in accuracy or remained within 1\% of its previous accuracy. The "Chest" measurement is now within the desired accuracy range of 25\% with a value of 24.13\%. The "Waist" measurement also greatly improved, however, is still narrowly outside the desired range with an accuracy of 25.11\%   

% Table generated by Excel2LaTeX from sheet 'Overall'
\begin{table}[htbp]
	\centering
	\caption{Results of the average accuracy of Upper Body Limbs after adjustments for clothing}
	\begin{tabularx}{\textwidth}{|Y|Y|Y|Y|Y|Y|}
		\toprule
		\textit{\textbf{Volunteer Number}} & \textit{\textbf{Chest}} & \textit{\textbf{Upper Left Arm}} & \textit{\textbf{Lower Left Arm}} & \textit{\textbf{Upper Right Arm}} & \textit{\textbf{Lower Right Arm}} \\
		\midrule
		\textit{\textbf{2}} & 27.84\% & 23.16\% & 9.84\% & 17.74\% & 13.24\% \\
		\midrule
		\textit{\textbf{3}} & 14.58\% & 10.99\% & 24.14\% & 7.33\% & 12.27\% \\
		\midrule
		\textit{\textbf{6}} & 23.39\% & 6.64\% & 15.91\% & 8.46\% & 8.63\% \\
		\midrule
		\textit{\textbf{7}} & 30.70\% & 12.34\% & 12.71\% & 12.02\% & 7.95\% \\
		\midrule
		\textit{\textbf{Total Avg Error}} & \textit{\textbf{24.13\%}} & \textit{\textbf{13.28\%}} & \textit{\textbf{15.65\%}} & \textit{\textbf{11.39\%}} & \textit{\textbf{10.52\%}} \\
		\bottomrule
	\end{tabularx}%
	\label{tab:upperBodyResultsClothingAdj}%
\end{table}%

% Table generated by Excel2LaTeX from sheet 'Overall'
\begin{table}[htbp]
	\centering
	\caption{Results of the average accuracy of Lower Body Limbs after adjustments for clothing}
	\begin{tabularx}{\textwidth}{|Y|Y|Y|Y|Y|Y|}
		\toprule
		\textit{\textbf{Volunteer Number}} & \textit{\textbf{Waist}} & \textit{\textbf{Upper Left Leg}} & \textit{\textbf{Lower Left Leg}} & \textit{\textbf{Upper Right Leg}} & \textit{\textbf{Lower Right Leg}} \\
		\midrule
		\textit{\textbf{2}} & 25.36\% & 15.32\% & 14.18\% & 13.21\% & 6.61\% \\
		\midrule
		\textit{\textbf{3}} & 20.51\% & 13.09\% & 24.62\% & 20.01\% & 10.12\% \\
		\midrule
		\textit{\textbf{6}} & 35.04\% & 20.78\% & 43.78\% & 9.58\% & 17.38\% \\
		\midrule
		\textit{\textbf{7}} & 19.53\% & 23.78\% & 22.77\% & 26.17\% & 16.17\% \\
		\midrule
		\textit{\textbf{Total Avg Error}} & \textit{\textbf{25.11\%}} & \textit{\textbf{18.24\%}} & \textit{\textbf{26.34\%}} & \textit{\textbf{17.24\%}} & \textit{\textbf{12.57\%}} \\
		\bottomrule
	\end{tabularx}%
	\label{tab:lowerBodyResultsClothingAdj}%
\end{table}%


\subsection{Other Empirical Observations}



8) Empirical Insights - Trying to determine a correct measurement, Legs not together, length of people, skinny person, skeleton not perfectly fitting, background edges not perfect \\

5) Uncertainty model\\

6) Circumference results - Ellipse
7) Circumference results - Rectangle


9) Improved UI
10) Data Set Analysis\\

Test of Table \ref{tab:testTable} added to results

% Table generated by Excel2LaTeX from sheet 'Volunteer 1'
\begin{table}[htbp]
	\centering
	\caption{Add caption}
	%\begin{adjustbox}{width=\textwidth}	
	%\resizebox{\textwidth}{!}{
	\begin{tabularx}{\textwidth}{|X|X|X|X|X|X|X|X|X|X|X|}
		\toprule
		& Chest & Waist & Upper Left Arm & Lower Left Arm & Upper Right Arm & Lower Right Arm & Upper Left Leg & Lower Left Leg & Upper Right Leg & Lower Right Leg \\
		\midrule
		\rowcolor[rgb]{ .573,  .816,  .314} Front & 38.92 & 38.27 & 7.97  & 9.28  & 7.09  & 7.27  & 24.72 & 12.55 & 24.38 & 12.02 \\
		\midrule
		\rowcolor[rgb]{ 0,  .69,  .941} Front & 38    & 31    & 11.5  & 9     & 12    & 7.5   & 15.5  & 10.5  & 15.5  & 10.5 \\
		\midrule
		Error & 2.42\% & 23.45\% & -30.70\% & 3.11\% & -40.92\% & -3.07\% & 59.48\% & 19.52\% & 57.29\% & 14.48\% \\
		\midrule
		\rowcolor[rgb]{ .573,  .816,  .314} Left  & 28.73 & 31.96 & 7.39  & 8.73  & \#N/A & \#N/A & 20.05 & 12.55 & \#N/A & \#N/A \\
		\midrule
		\rowcolor[rgb]{ 0,  .69,  .941} Left  & 20    & 20.5  & 11    & 7.5   & \#N/A & \#N/A & 14.5  & 11.5  & \#N/A & \#N/A \\
		\midrule
		Error & 43.65\% & 55.90\% & -32.82\% & 16.40\% & \#N/A & \#N/A & 38.28\% & 9.13\% & \#N/A & \#N/A \\
		\midrule
		\rowcolor[rgb]{ .573,  .816,  .314} Back  & 80.37 & 34.58 & 8.12  & 8.31  & 7.88  & 7.37  & 22.38 & 12.89 & 23.54 & 11.05 \\
		\midrule
		\rowcolor[rgb]{ 0,  .69,  .941} Back  & 35    & 30    & 11    & 7.5   & 11.5  & 9.5   & 15    & 11.5  & 14.5  & 11.5 \\
		\midrule
		Error & 129.63\% & 15.27\% & -26.18\% & 10.80\% & -31.48\% & -22.42\% & 49.20\% & 12.09\% & 62.34\% & -3.91\% \\
		\midrule
		\rowcolor[rgb]{ .573,  .816,  .314} Right & 25.68 & 29.42 & \#N/A & \#N/A & 9.8   & 8.78  & \#N/A & \#N/A & 17.38 & 24.58 \\
		\midrule
		\rowcolor[rgb]{ 0,  .69,  .941} Right & 21    & 22    & \#N/A & \#N/A & 10.5  & 9.5   & \#N/A & \#N/A & 17    & 11.5 \\
		\midrule
		Error & 22.29\% & 33.73\% & \#N/A & \#N/A & -6.67\% & -7.58\% & \#N/A & \#N/A & 2.24\% & 113.74\% \\
		\bottomrule
	\end{tabularx}%
	%}
	%\end{adjustbox}
	\label{tab:testTable}%
\end{table}%

\section{3D Modelling}

\section{User Interface Observations}
