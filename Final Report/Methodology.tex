\chapter{Methodology} \label{methodology}

This is what I did to test and confirm my hypothesis.


You may want to split this chapter into sub chapters depending on your design. I suggest you change
the title to something more specific to your project.

This is where you describe your design process in detail, from component/device selection to actual
design implementation, to how you tested your system. Remember detail is important in technical
writing. Do not just write I used a computer give the computer specifications or the oscilloscopes part
number. Describe the system in enough detail so that someone else can replicate your design as well
as your testing methodology.

If you use or design code for your system, represent it as flow diagrams in text.

\begin{enumerate}
\item This is a bullet point test
\item I hope this works
	
\end{enumerate}	

\section{Aim} \label{methodologyAim}
The aim of this project was to create a system that enabled the measurement and 3D modelling of different parts of a human body through a well designed user interface.

\section{Hypothesis} \label{methodologyHypothesis}
To determine if the above aim is met, the following three hypotheses, each with their own unique sub-hypotheses, will be tested:

\begin{enumerate}
	\item The extremities of a human body can be measured and perform within the following criteria:
	\begin{enumerate}
		\item The overall measurement of the extremities of a human should perform within an average accuracy tolerance of 25\%.
		\item Each measurement "view" should perform within an average accuracy tolerance of 25\%.
		\item Each individual limb measurement should perform within an average accuracy tolerance of 25\%.
		\item The presence of loose clothing should not affect the above performance criteria by more than 10\%.
		\item The system should be robust and able to accurately detect the human body and take all necessary measurements. 
		\item Each reading should have an uncertainty of less than 10\%. 
	\end{enumerate}
	\item The measured extremities of the human body can be used to create 3D models of relevant body parts: 
	\begin{enumerate}
		\item The circumference of modelled body parts should perform within an average accuracy tolerance of 25\%.
		\item Using an ellipse to model human body parts will yield more accurate results than using a rectangle model. 
	\end{enumerate}
	\item The user interface should allow for the successful processing of a person: 
	\begin{enumerate}
		\item The user interface should be robust and limit the affect that user mistakes have on the measurement process.
		\item The user interface should be easy to use and as autonomous as possible.
	\end{enumerate}
\end{enumerate}
